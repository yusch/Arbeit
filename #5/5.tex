\documentclass[dvipdfmx]{jsarticle}
\usepackage{url}
\usepackage{listings}
\usepackage[dvipdfmx]{graphicx}
\usepackage{here}
\usepackage{ascmac}\usepackage{listings, jlisting, color}
\definecolor{OliveGreen}{rgb}{0.0,0.6,0.0}
\definecolor{Orenge}{rgb}{0.89,0.55,0}
\definecolor{SkyBlue}{rgb}{0.28, 0.28, 0.95}
\lstset{
  language={C++}, % 言語の指定
  basicstyle={\ttfamily},
  identifierstyle={\small},
  commentstyle={\smallitshape},
  keywordstyle={\small\bfseries},
  ndkeywordstyle={\small},
  stringstyle={\small\ttfamily},
  frame={tb},
  breaklines=true,
  columns=[l]{fullflexible},
  numbers=left,
  xrightmargin=0zw,
  xleftmargin=3zw,
  numberstyle={\scriptsize},
  stepnumber=1,
  numbersep=1zw,
  lineskip=-0.5ex,
  keywordstyle={\color{SkyBlue}},     %キーワード(int, ifなど)の書体指定
  commentstyle={\color{OliveGreen}},  %注釈の書体
  stringstyle=\color{Orenge}          %文字列
}

\begin{document}
\title{課題5}
\author{ロボ団マスターコース}
\maketitle

\section{はじめに}
実機を使って動かしてみよう!\\
今回は、音について勉強しよう!\\
まずは、下のコードを写経しよう。どんな動きをするか確認できたら、自分の曲を作ってみよう!\\
\section{ソースコード}
どんな音楽か聞いてみよう!
\begin{lstlisting} 
from microbit import *
import music
notes = ['c4:1', 'e', 'g', 'c5', 'e5', 'g4', 'c5', 'e5', 'c4', 'e', 'g', 'c5', 'e5', 'g4', 'c5', 'e5']
music.play(notes)
\end{lstlisting}
microbitにははじめから曲が用意されているよ!\\
聞いてみよう!
\begin{lstlisting} 
import music
music.play(music.FUNK)
\end{lstlisting}
FUNKを下の曲名に変えてみよう\\
\> \textbf{BIRTHDAY}\\
\> \textbf{DADADADUM}\\

音楽のテンポを設定する関数
\begin{lstlisting} 
music.set_tempo(ticks=4, bpm=120)
\end{lstlisting}
音楽の再生する関数
\begin{lstlisting} 
music.play(music, pin=microbit.pin0, wait=True, loop=False)
\end{lstlisting}
\end{document}
